\chapter*{Abstract}

The proliferation of Internet of Things (IoT) nodes and biomedical sensors has created a stringent demand for ultra-low-power signal processing interfaces. In many of these applications, the signals of interest are sparse or bursty, characterized by long periods of inactivity. Classical synchronous Analog-to-Digital Converters (ADCs) are inherently inefficient in such scenarios, as they consume dynamic power continuously due to the global clock, regardless of the input signal activity.

This dissertation proposes the design and implementation of an Asynchronous Flash ADC. By removing the clock, the proposed architecture aligns power consumption with the input signal activity, theoretically achieving near-zero power dissipation during idle periods. However, the removal of the clock introduces design challenges, particularly regarding the accuracy of continuous-time comparators, which are prone to offset voltages due to transistor mismatch in scaled CMOS technologies.

To address this, an offline trimming strategy is proposed to calibrate the comparator offsets without compromising the high-speed operation of the flash topology. The work encompasses the theoretical analysis, schematic design, and validation of the system through Analog/Mixed-Signal (A/MS) co-simulation. The expected outcome is a robust, clockless ADC architecture that offers a superior Figure of Merit (FoM) for sparse signal applications compared to traditional synchronous counterparts.

\textbf{Keywords:} Analog-to-Digital Converter, Asynchronous Design, Flash ADC, Level-Crossing Sampling, Offset Trimming, Low Power, IoT.