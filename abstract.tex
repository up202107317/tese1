\chapter*{Abstract}

The proliferation of Internet of Things and biomedical sensors has created a high demand for low-power signal processing interfaces. In many of these applications, the signals of interest are sparse, characterized by long periods of inactivity. Classical synchronous Analog-to-Digital Converters are inherently inefficient in such scenarios, as they consume dynamic power continuously due to the global clock, regardless of the input signal activity.

This dissertation proposes the design and implementation of an Asynchronous Flash ADC. By removing the clock, the proposed architecture aligns power consumption with the input signal activity, theoretically achieving a much lower power consuption. However, the removal of the clock introduces design challenges (acrescentar aqui problemas)

To address this, an offline trimming strategy is proposed to calibrate the comparator offsets without compromising the high-speed operation of the flash topology. The work encompasses the theoretical analysis, schematic design, and validation of the system through Analog/Mixed-Signal co-simulation. The expected outcome is a robust, clockless ADC architecture that offers a superior Figure of Merit for sparse signal applications in compared to conventional synchronous architectures.

\textbf{Keywords:} Analog-to-Digital Converter, Asynchronous Design, Flash ADC, Level-Crossing Sampling, Offset Trimming, Low Power, IoT.